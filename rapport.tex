\documentclass[10pt,a4paper,danish]{article}
%% Indlæs ofte brugte pakker
\usepackage{amssymb}
\usepackage[danish]{babel}
\usepackage[utf8]{inputenc}
\usepackage{listings}
\usepackage{fancyhdr}
\usepackage{hyperref}
\usepackage{booktabs}
\usepackage{graphicx}

\pagestyle{fancy}
\fancyhead{}
\fancyfoot{}
\rhead{\today}
\rfoot{\thepage}
\setlength\parskip{0em}
\setlength\parindent{0em}

%% Titel og forfatter
\title{G-opgave \\Oversættere\\Vinter 2011/2012}
\author{Naja Wulff Mottelson (vsj465) \\
        Søren Egede Pilgård (vpb984) \\
        Mark Schor (vcl329)}

%% Start dokumentet
\begin{document}

%% Vis titel
\maketitle
\newpage

%% Vis indholdsfortegnelse
\tableofcontents
\newpage

%% Rapport, baby!
\section{Indledning}
På nuværende hvor langt er vi nået med opgaven? Hvad virker, hvad virker ikke?

\section{Lexer og parser}
I vores arbejde med lexeren har vi indledningsvist tilføjet de manglende
nøgleord (while og char), samt fjernet nøgleordet then, eftersom
dette ikke er til stede i den udleverede 
grammatik for 100. Vi har herefter oprettet
regulære udtryk i reglen Token  til at matche tilføjelserne (som ligeledes
tæller ==-operatren, return-sætninger,referencer, krølleparenteser samt
firkantede parenteser). 

\paragraph{}
I arbejdet med parseren har vi genemgået den udleverede grammatik 
og indføjet de manglende termer som tokens og types i Parser.grm
(hvor vi i samme omgang har tilføjet præcedens og associativitet - 
hovedsageligt ved at tilføje venstreassociativitet 
til ==-operatoren).

\paragraph{}
Ved kørsel af compile.sh modtager vi ingen fejlbeskeder eller 
notifikationer om shift/reduce konflikter. Eftersom vi antager at 
parsergeneratoren er korrekt implementeret, konkluderer vi derfor at den 
grammatik vi parser er entydig. 

\section{Typetjekker}
Hoveddelen af det arbejde vi har lagt i typetjekkeren har været i form
af udvidelser i funktionen Type.checkExp og Type.checkStat.
Udover dette har vi dog indledningsvist tilføjet 100s indbyggede
funktioner (balloc, walloc, getstring og putstring) i Type.checkProg, samt 
tilføjet typerne for 100s abstrakte syntaks (Char, CharRef, IntRef).
For at undgå unødig mønstergenkendelse (både i typetjekkeren og 
i Compiler.sml) har vi introduceret hjælpefunktionen Type.ignoreChar, 
som konverterer tegn til heltal: 

\begin{verbatim}
  fun ignoreChar (Char) = Int
    | ignoreChar ty = ty
\end{verbatim}

\subsection{Tjek af erklæringer}
Vi har udvidet en masse 

 
\subsection{Tjek af udtryk}
I Type.checkExp har vi tilføjet mønstre for de manglende 100-typer (CharConst, 
StringConst, Equal), samt tilføjet overlæsning at plus- og minusoperatorerne
så at de er i stand til at tage både heltal, tegreferencer og heltalsreferencer
som operander.  

\subsection{Tjek af sætninger}
Idet 100 indeholder en del kontrolstrukturer som kan skabe usikkerhed om 
hvorvidt en funktion vil returnere, er dette ikke som sådan et trivielt problem.
Under typetjekket kan man f. eks. møde en lignende funktion:\\\\

\begin{verbatim}
int foo(int bar)
    if 1
        return bar
\end{verbatim}

Funktionen foo er syntaktisk korrekt, men indeholder 
en typefejl eftersom det ikke er sikkert at return-sætningen vil
blive nået. Der findes adskillige måder at tjekke for denne slags typefejl -
en metode kunne f.eks. være at undersøge en funktions sidste linje og 
kontrollere at denne altid vil returnere. Denne løsning vil dog
træffe en problematik ifbm. if-else-konstruktionen, da denne
kan have return-sætninger i begge sine tilfælde. Således  vil enhver
funktion indeholdende en sådan if-else altid være sikker på at returnere, 
ligegyldigt hvad der står i dens sidste linje kode.

\paragraph{}
Vi implementerer tjekket ved at tilføje et yderligere output i Type.checkStat - 
en sandhedsværdi som angiver hvorvidt den matchede sætning indeholder
en return-sætning. I tilfældene if og while returneres falsk,
eftersom begge konstruktioner vil kræve en return-sætning i en følgende
blok. I tilfældet if-else kontrollerer vi at begge tilfælde returnerer og 
sender denne sandhedsværdi videre. Type.checkStats output kontrolleres senere
i Type.checkFunDec, hvor en fejl kastes i tilfælde af manglende returns. 

\paragraph{}
Som nævnt skal det dog også kontrolleres at return-sætningens og funktionens
type er kongruente, til hvilket vi har overvejet forskellige løsninger. Den
mest funktionelle løsning forekommer at være at lade Type.checkFunDec returnere
funktionens returtype. Type.checkStat vil herfter kunne udføre sit vanlige
tjek af tilstedeværelsen af return-sætninger, og herefter returnere enten NONE
i tilfælde af manglende returns eller SOME af sætningens returtype, samt
sætningens position. 
Sammenligningen af funktionens og return-sætningens typer (samt tjekket
for manglende returns) vil herefter kunne foretages i Type.checkFunDec.
Denne implementering bliver dog problematisk så 
snart forgreninger i koden medfører flere return-sætninger i samme blok.
Type.checkStat
ville i dette tilfælde være nødt til at returnere en liste indeholdende
tupler af positioner og returtyper, som ville skulle gennemsøges i 
Type.checkFunDec for at udføre typesammenligningen. 

\paragraph{}
Den måde vi har valgt at implementere tjekket på er (ligesom i algoritmen 
nævnt ovenfor) ved at sende en funktions returtype samt position med som 
output fra Type.checkFunDec, hvorefter kontrollen af typekongruensen foretages
i Type.checkStat. 

%%NB: Det faktum at vi kan forudsætte prædikaterne til at være enten 
%%sande eller falske redder os fra Rice's sætning. 

\subsection{Tjek af venstreværdier}
Vi har udvidet Type.checkLval med tilfældene S100.Deref samt
S100.Lookup, så at det bliver muligt at typetjekke forsøg på 
at indeksere arrays (eller rettere repræsentationen af arrays
i form af en reference). Essentielt kan to fejl opstå i denne 
situation: det indekserede array kan være udefineret, eller 
indekset kan være af anden type end heltal. I vores implementering
har vi valgt at prioritere den første type fejl højest, så at 
en fejlbesked vedr. eksistensen af referencen vil blive kastet
førend en om indeks. I tilfældet af opslag (Deref eller Lookup)har vi 
endvidere ændret Compile.checkLvals returtype til at returnere 
den type referencen peger på, i stedet for hhv. Int- eller CharRef. 
Implementeringen er følgende (bemærk at Type.checkLval er gensidigt
rekursiv med Type.checkExp, og derfor erklæres med and): 

\begin{verbatim}

\end{verbatim} 

\section{Kodegenerering og oversætter}
Vi har indsat de indbyggede funktioner (udover putint og getint) i 
Compiler.compile. 

Eftersom det er muligt at have deklarationer i starten af blokke, er det 
nødvendigt at kunne udvide symboltabellen for at implementere dem i 
oversætteren. Vi har derfor introduceret funktionerne compileDecs og 
extend, som begge er hentet fra Type.sml - extend er dog 
en let modificeret version af Type.extend  som også
opretter en ny lokation.  

\subsection{Håndtering af tildelinger}
Et pudsigt tilfælde ved tildelingsoperatoren er dens opførsel ved 
simultane tildelinger af tegn og heltal. Et eksempel på dette kunne
være følgende blok: 

\begin{verbatim}
{
int i; char *c;
i = c = 256;
i == 0;
c == 0;
}
\end{verbatim}

Eftersom antallet af allokerede bits varierer for heltal og tegn, 
(og det kun er den niende bit der sat i den binære repræsentation
af tallet 256) vil c skulle sættes til 00000000 i tildelingen
i = c = 256;. I den oprindelige version af Compiler.compileExp
bliver variablen place gemt ved  tildeling af tegnreferencer - 
her vil c således blive sat korrekt, men i vil blive sat til 
256 grundet de ekstra bits i place den kan registrere. 

\paragraph{}
Vi har løst dette problem ved at trunkere place (reelt: t, eftersom
place tidligere bliver flyttet over i t), således at det kun er de
otte mindst betydende bits der bliver brugt til at gemme referencen. 
Se nedenfor for implementeringen (i Compiler.checkExp):

\begin{verbatim}

\end{verbatim}

\subsection{Håndtering af referencer}
Vi har valgt at udvide datatypen Location med typen Mem, som repræsenterer
referencer (som ligger i hukommelsen og ikke i registerbanken). 

Vi har skrevet funktionen ignoreChars, fordi vi abstraherer os fra at skulle 
håndtere char-typen i typetjekkeren.    
 
Vi har udvidet Compiler.compileLval til at returnere 

\section{Afprøvning}
Vi har udvidet med en test for manglende main-funktioner. 

Samt en test for 

\section{Opsummering}

\end{document}