
\documentclass[10pt,a4paper,danish]{article}
%% Indlæs ofte brugte pakker
\usepackage{amssymb}
\usepackage[danish]{babel}
\usepackage[utf8]{inputenc}
\usepackage{listings}
\usepackage{fancyhdr}
\usepackage{hyperref}
\usepackage{booktabs}
\usepackage{graphicx}

\pagestyle{fancy}
\fancyhead{}
\fancyfoot{}
\rhead{\today}
\rfoot{\thepage}
\setlength\parskip{0em}
\setlength\parindent{0em}

%% Titel og forfatter
\title{G-opgave \\Oversættere\\Vinter 2011/2012}
\author{Naja Wulff Mottelson (vsj465) \\
        Søren Egede Pilgård (vpb984) \\
        Mark Schor (vcl329)}

%% Start dokumentet
\begin{document}

%% Vis titel
\maketitle
\newpage

%% Vis indholdsfortegnelse
\tableofcontents
\newpage

%% Rapport, baby!
\section{Indledning}
Nærværende rapport tjener som dokumentation af vores arbejde med G-opgaven
i kurset Oversættere. Vi har her implementeret lexer, parser, typetjekker og
oversætter for sproget 100. På nuværende tidspunkt er vi i stand til at 
køre samtlige udleverede fejlbehæftede programmer med de forventede fejl. 
Vi kan derudover oversætte et program der tester for tilstedeværelse af 
en main-funktion (se Appendix), samt oversætte
og køre de udleverede korrekte testprogrammer uden at modtage fejl. 

\section{Lexer og parser}
I vores arbejde med lexeren har vi indledningsvist tilføjet de manglende
nøgleord (\texttt{while} og \texttt{char}) samt fjernet nøgleordet then, eftersom
dette ikke er 
til stede i den udleverede 
grammatik for 100. Vi har herefter oprettet
regulære udtryk i reglen Token  til at matche tilføjelserne (som ligeledes
tæller ==-operatoren, return-sætninger,referencer, krølleparenteser samt
firkantede parenteser). 

\paragraph{}
I arbejdet med parseren har vi genemgået den udleverede grammatik 
og indføjet de manglende termer som tokens og types i \texttt{Parser.grm}
(hvor vi i samme omgang har tilføjet præcedens og associativitet - 
hovedsageligt ved at tilføje venstreassociativitet 
til ==-operatoren).

\paragraph{}
Ved kørsel af \texttt{compile.sh} modtager vi ingen fejlbeskeder eller 
notifikationer om shift/reduce konflikter. Eftersom vi antager, at 
parsergeneratoren er korrekt implementeret, konkluderer vi derfor at den 
grammatik, vi parser, er entydig. 

\section{Typetjekker}
Hoveddelen af det arbejde vi har lagt i typetjekkeren har været i form af
udvidelser i funktionen \texttt{Type.checkExp} og \texttt{Type.checkStat}.
 Udover dette har
vi dog indledningsvist tilføjet 100s indbyggede funktioner (\texttt{balloc}, 
\texttt{walloc}, \texttt{getstring} og \texttt{putstring}) i 
\texttt{Type.checkProg}, samt tilføjet typerne for 100s abstrak-
te syntaks (\texttt{Char, CharRef, IntRef}).

\subsection{Hjælpefunktioner i typetjekkeren}
I typetjekkeren har vi udvidet en række hjælpefunktioner, som beskrives nedenfor:  
\subsubsection{ignoreChar} 
For at undgå unødig mønstergenkendelse
(både i typetjekkeren og i \texttt{Compiler.sml}
) har vi introduceret hjælpefunktionen
\texttt{Type.ignoreChar}, som konverterer tegn til heltal:

\begin{verbatim}
  fun ignoreChar (Char) = Int
    | ignoreChar ty = ty
\end{verbatim}

\subsubsection{getType}
For at kunne implementere referencer har vi udvidet \texttt{Type.getType}
fra udelukkende at kalde \texttt{Type.convertType} på sit input til at 
angive 

\subsubsection{mismatch og typeToString}
Begge disse funktioner benyttes udelukkende i forbindelse
med aflusning af koden, hvor de sørger for at brugeren 
kan få nogenlunde sigende fejlbeskeder ved fejlagtig sammensætning
af typer. 

\begin{verbatim}

  fun typeToString Int = "Int"
    | typeToString Char = "Char"
    | typeToString IntRef = "IntRef"
    | typeToString CharRef = "CharRef"

  fun mismatch t1 t2 = (typeToString t1) ^ " != " ^ (typeToString t2)

\end{verbatim}

\subsection{Tjek af udtryk}
I \texttt{Type.checkExp} har vi tilføjet mønstre for de manglende 100-typer
 (\texttt{CharConst, StringConst, Equal}) samt tilføjet overlæsning af 
plus- og minusoperatorerne,
så at de er i stand til at tage både heltal, tegreferencer og heltalsreferencer
som operander.  

\paragraph{}
Eftersom der kan blive erkæret nye variable i starten af en blok har vi brug 
for at kunne opdatere symboltabellen for at kunne typetjekke blokke. Vi har
derfor udvidet \texttt{Type.checkStat} til at tage den eksisterende
\texttt{vtable} som input og generere en ny symboltabel som konkateneres
 med den gamle i tilfælde af blokke. 

\subsection{Tjek af sætninger}
Idet 100 indeholder en del kontrolstrukturer, som kan skabe usikkerhed, om 
hvorvidt en funktion vil returnere, er dette ikke som sådan et trivielt problem.
Under typetjekket kan man f. eks. møde en lignende funktion:\\\\

\begin{verbatim}
int foo(int bar)
    if 1
        return bar
\end{verbatim}

Funktionen foo er syntaktisk korrekt men indeholder 
en typefejl, eftersom det ikke er sikkert, at return-sætningen vil
blive nået. Der findes adskillige måder at tjekke for denne slags typefejl -
en metode kunne f.eks. være at undersøge en funktions sidste linje, og 
kontrollere at denne altid vil returnere. Denne løsning vil dog
træffe en problematik ifbm. \texttt{if-else}-konstruktionen, da denne
kan have return-sætninger i begge sine tilfælde. Således  vil enhver
funktion indeholdende en sådan \texttt{if-else}
 altid være sikker på at returnere, 
ligegyldigt hvad der står i dens sidste linje kode.

\paragraph{}
Vi implementerer tjekket ved at tilføje et yderligere output i 
\texttt{Type.checkStat} - 
en sandhedsværdi som angiver, hvorvidt den matchede sætning indeholder
en return-sætning. I tilfældene \texttt{if} og 
\texttt{while} returneres falsk,
eftersom begge konstruktioner vil kræve en return-sætning i en følgende
blok. I tilfældet \texttt{if-else}
 kontrollerer vi, at begge tilfælde returnerer og 
sender denne sandhedsværdi videre. \texttt{Type.checkStats}
 output kontrolleres senere
i \texttt{Type.checkFunDec}, 
hvor en fejl kastes i tilfælde af manglende returns. 

\paragraph{}
Som nævnt skal det dog også kontrolleres, at return-sætningens og funktionens
type er kongruente, til hvilket vi har overvejet forskellige løsninger. Den
mest funktionelle løsning forekommer at være at lade 
\texttt{Type.checkFunDec} returnere
funktionens returtype. \texttt{Type.checkStat}
 vil herfter kunne udføre sit vanlige
tjek af tilstedeværelsen af return-sætninger og herefter returnere enten 
\texttt{NONE}
i tilfælde af manglende returns eller \texttt{SOME}
 af sætningens returtype, samt
sætningens position. 
Sammenligningen af funktionens og return-sætningens typer (samt tjekket
for manglende returns) vil herefter kunne foretages i
\texttt{Type.checkFunDec}.
Denne implementering bliver dog problematisk, så 
snart forgreninger i koden medfører flere return-sætninger i samme
\texttt{Type.checkStat}
ville i dette tilfælde være nødt til at returnere en liste indeholdende
tupler af positioner og returtyper, som ville skulle gennemsøges i 
\texttt{Type.checkFunDec} for at udføre typesammenligningen. 

\paragraph{}
Den måde, vi har valgt at implementere tjekket på, er (ligesom i algoritmen 
nævnt ovenfor) ved at sende en funktions returtype samt position med som 
output fra \texttt{Type.checkFunDec}, hvorefter kontrollen af typekongruensen
foretages i \texttt{Type.checkStat}: 

\begin{verbatim}

  fun checkStat s vtable ftable returntype =
    case s of
      S100.EX e => (checkExp e vtable ftable; false)
    | S100.If (e,s1,p) =>
        if checkExp e vtable ftable = Int
	then (checkStat s1 vtable ftable returntype; false)
	else raise Error ("Condition should be integer",p)
    | S100.IfElse (e,s1,s2,p) =>
      let
        val r1 = checkStat s1 vtable ftable returntype
        val r2 = checkStat s2 vtable ftable returntype
      in
        if checkExp e vtable ftable = Int
        then r1 andalso r2
	else raise Error ("Condition should be integer",p)
      end
    | S100.While (e,s,p) =>
      if checkExp e vtable ftable = Int
      then (checkStat s vtable ftable returntype; false)
      else raise Error ("Condition should be integer",p)
    | S100.Return (e,p) => if checkExp e vtable ftable = returntype
                           then true
                           else raise Error ("Returned value not of "^
                                             "functions returntype: "^mismatch
                                              returntype (checkExp e vtable 
                                              ftable) ,p)

    | S100.Block ([],[],p) => false
    | S100.Block ([], s::ss,p) =>
      let
        val r1 = checkStat s vtable ftable returntype
        val r2 = checkStat (S100.Block ([], ss, p)) vtable ftable returntype
      in
        r1 orelse r2
      end
    | S100.Block (ds, ss,p) =>
      let
        val vtable' = checkDecs ds
      in
        checkStat (S100.Block([], ss, p)) (vtable' @ vtable) ftable returntype
      end

  fun checkFunDec (t,sf,decs,body,p) ftable =
        if checkStat body (checkDecs decs) ftable (getType t sf)
        then ()
        else raise Error ("Function needs valid return statements",p)
\end{verbatim} 

%%NB: Det faktum at vi kan forudsætte prædikaterne til at være enten 
%%sande eller falske redder os fra Rice's sætning. 

\subsection{Tjek af venstreværdier}
Vi har udvidet \texttt{Type.checkLval} med tilfældene 
\texttt{S100.Deref} samt \texttt{S100.Lookup}, så at det
 bliver muligt at typetjekke forsøg på 
at indeksere arrays (eller rettere repræsentationen af arrays
i form af en reference). Essentielt kan to fejl opstå i denne 
situation: det indekserede array kan være udefineret, eller 
indekset kan være af anden type end heltal.
 I tilfældet af opslag (\texttt{Deref} eller 
\texttt{Lookup})har vi 
endvidere ændret \texttt{Type.checkLvals} returtype, til at returnere 
den type referencen peger på, i stedet for hhv. \texttt{Int-} eller
 \texttt{CharRef}. 
Implementeringen er følgende (bemærk at \texttt{Type.checkLval} er gensidigt
rekursiv med \texttt{Type.checkExp} og derfor erklæres med \texttt{and}): 

\begin{verbatim}
  and checkLval lv vtable ftable =
    case lv of
      S100.Var (x,p) =>
        (case lookup x vtable of
	   SOME t => t
	 | NONE => raise Error ("Unknown variable: "^x,p))
    | S100.Deref (x,p) =>
      (case lookup x vtable of
         SOME IntRef => Int
       | SOME CharRef => Char
       | SOME _ => raise Error ("You can not use a non-reference type as an address!", p)
       | NONE   => raise Error ("Unknown variable: "^x,p))
    | S100.Lookup (x,e,p) =>
      if ignoreChar (checkExp e vtable ftable) = Int
      then
        (case lookup x vtable of
           SOME IntRef => Int
         | SOME CharRef => Char
         | SOME _ => raise Error ("You can not use a non-reference type as an address!", p)
         | NONE   => raise Error ("Unknown variable: "^x,p))
      else raise Error ("Index not a number",p)

\end{verbatim} 

\section{Kodegenerering og oversætter}
Eftersom det er muligt at have deklarationer i starten af blokke, er det 
nødvendigt at kunne udvide symboltabellen for at implementere dem i 
oversætteren. Vi har derfor introduceret funktionerne
 \texttt{Compile.rcompileDecs}
 og \texttt{Compiler.extend}, som begge er hentet fra \texttt{Type.sm} - 
\texttt{Compiler.extend} er dog 
en let modificeret version af \texttt{Type.extend}, som også
opretter en ny lokation.  


\subsection{Håndtering af tildelinger}
Et pudsigt tilfælde ved tildelingsoperatoren er dens opførsel ved 
simultane tildelinger af tegn og heltal. Et eksempel på dette kunne
være følgende blok: 

\begin{verbatim}
{
int i; char *c;
i = c = 256;
i == 0;
c == 0;
}
\end{verbatim}

Eftersom antallet af allokerede bits varierer for heltal og tegn, 
og det kun er den niende bit der sat i den binære repræsentation
af tallet 256, vil \texttt{c} skulle sættes til 00000000 i tildelingen
\texttt{i = c = 256};. I den oprindelige version af 
\texttt{Compiler.compileExp} bliver variablen \texttt{place}
 gemt ved tildeling af tegnreferencer - 
her vil \texttt{c} således blive sat korrekt, men \texttt{i}
 vil blive sat til 
256 grundet de ekstra bits i \texttt{place} den kan registrere.
Vi har løst dette problem ved at afkorte \texttt{place} (reelt: 
\texttt{t}, eftersom \texttt{place}
tidligere bliver flyttet over i \texttt{t}), således at det kun er
 de otte mindst betydende
bits der bliver brugt til at gemme referencen. Se nedenfor for
 implementeringen
(i \texttt{Compiler.checkExp}):

\begin{verbatim}

    | S100.Assign (lval,e,p) =>
        let
          val t = "_assign_"^newName()
	  val (code0,ty,loc,_) = compileLval lval vtable ftable
	  val (_,code1) = compileExp e vtable ftable t
	in
          
	  case (ty,loc) of

            (tty, Reg x) =>
              (tty,
               code0 @ code1 @ [Mips.MOVE (x,t), Mips.MOVE (place,t)])
          | (Type.Int, Mem x) =>
            (Type.Int,
             code0 @ code1 @
             [Mips.SW (t,x,"0"), Mips.MOVE (place,t)])
          | (Type.Char, Mem x) =>
            (Type.Char,
               code0 @ code1 @
               [Mips.ANDI(t,t,"255"), Mips.SB (t,x,"0"), Mips.MOVE (place,t)])
          | (Type.IntRef, Mem x) => raise Error ("Type error, "^
                                                "asignment of"^
                                                "pointer-pointers "^
                                                "not implemented!", p)
          | (Type.CharRef, Mem x) => raise Error ("Type error, "^
                                                "asignment of"^
                                                "pointer-pointers "^
                                                "not implemented!", p)

	end

\end{verbatim}

\subsection{Implementering af referencer}
Til brug i vores implementering af referencer har vi valgt at
udvide datatypen \texttt{Location} med typen \texttt{Mem} til at
repræsentere referencer, som ligger i hukommelsen istedet for
registerbanken. 

Idet hovedformålet ved referencerne i 100 er at de kan indekseres, 
bliver denne udvidelse af \texttt{Location} hovedsageligt udnyttet 
i \texttt{Compiler.compileLval} og \texttt{Compiler.compileExp}
: Ved \texttt{Compiler.compileLval} vil funktionen, i tilfældet
 \texttt{S100.Lookup}, 
udregne det udtryk der benyttes til at indeksere referencen, 
og returnerer den kode der udregner forskydningen ift. referencens 
adresse samt referencens type og lokation. \texttt{Compiler.compileLval}
bliver kaldt i \texttt{Compiler.compileExp}, hvor funtionen skelner 
imellem hvorvidt lokationen angiver Reg eller Mem. I tilfælde af 
registre vil \texttt{Compiler.compileExp} ændre i adressen, 
i tilfælde af hukommelse vil den ændre i selve dataet på bemeldte
adresse. Det samme gælder for \texttt{Compiler.compileExp}s opførsel ved
tildeling.

\section{Afprøvning}
Vi har udvidet antallet af afprøvninger med en test for manglende main-funktioner ourerror01. Alle vores testprogrammer kan oversættes og afleverer de korrekte resultater ifølge vores afprøvninger. Vores afprøvningsdata er følgende:\\
\begin{itemize}
\item Charref er blevet testet med inputdataen: (0) (1) (2) og (4) (5) og gav outputtet (97) (98) (114) (99) (97), hvilket svarer overens med de rigtige bogstaver i ASCII-tegnsættet.
\item Copy er blevet testet med inputdataen: (Husk at teste jeres compiler ordentligt!) og fik outputtet (Husk at teste)
\item DFA er blevet testet med inputdataen: (1) (10) (11) (100) (1000) og fik outputtet (1) (2) (0) (1) (2)
\item Fib er blevet testet med inputdataen: (5) (6) (7) (20) og fik outputtet (5) (8) (13) (6765)
\item Sort er blevet testet med inputdataen: (9,6,8,2,4,7,5,4,3,9) og fik outputtet (2,3,4,4,5,6,7,8,9)
\item Sort2 er blevet testet med inputdataen: (9,6,8,2,4,7,5,4,3,9) og fik outputtet (2,3,4,4,5,6,7,8,9)
\item Ssort er blevet testet med inputdataen: (Husk at teste jeres compiler ordentligt!) og fik outputtet (!Hacdeeeeeegiijkllmnooprrrssstttttu)
\item Ssort2 er blevet testet med inputdataen: (Husk at teste jeres compiler ordentligt!) og fik outputtet (!Hacdeeeeeegiijkllmnooprrrssstttttu)
\item error1 gav fejlbeskeden: Redeclaration of main
\item error2 gav fejlbeskeden: Double declaration of x
\item error3 gav fejlbeskeden: Unknown function: getchar
\item error4 gav fejlbeskeden: Redeclaration of putint
\item error5 gav fejlbeskeden: Type mismatch in addition at line
\item error6 gav fejlbeskeden: Type mismatch in substraction
\item error7 gav fejlbeskeden: Type mismatch in substraction
\item error8 gav fejlbeskeden: Can't compare different types
\item error9 gav fejlbeskeden: Can't compare different types
\item error10 gav fejlbeskeden: Arguments don't match declaration of f
\item error11 gav fejlbeskeden: Arguments don't match declaration of f
\item error12 gav fejlbeskeden: Bad string!
\item error13 gav fejlbeskeden: Function needs valid return statements
\item error14 gav fejlbeskeden: Function needs valid return statements
\item error15 gav fejlbeskeden: Returned value not of functions returntype: Int != CharRef
\end{itemize}




\section{Appendix}
\subsection{Test af tilstedeværelse af main-funktioner}
Error-programmet som vi har benyttet til at afprøve for 
tilstedeværelse af main-funktioner er implementeret således: 

\begin{verbatim}
\end{verbatim}
/* ourerror01.100 */
/* Missing main function */

int foo()
{
        return 1;
}
\end{document}
